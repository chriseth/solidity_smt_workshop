\section{SMT Encoding}
\label{section:smt}

\subsection{Implemented (or almost there)}

The SMT encoding is computed in a top-down way traversing the AST of the
Solidity program.  The context regarding the SMT solver, contract storage, and
local variables of functions is cleared before each function of a contract is
visited.

Inside each function, a variable declaration leads to a correspondent SMT
variable that is assigned the default value of the declared type.
%
Function parameters are initialized with a range of valid values for the given
type, since their value is unknown.  For instance, a parameter \code{uint32 x}
is initialized as $0 \le x < 2^{32}$ (32 bits), and a parameter \code{address
a} is assigned the range $0 \le a < 2^{(8*20)}$ (20 bytes).
%
The encoding of variable assignment follows the \emph{Single Static Assignment}
(SSA) where each assignment to a program variable introduces a new SMT variable
that is assigned to only once.
%
The encoder currenty supports \emph{Bool} and the various sizes of
\emph{Integer} variables.

Because the Solidity code is visited via its AST, we can maintain a stack of
conditions that lead to the piece of code being analyzed: whenever an
\emph{if-statement} is visited its condition is pushed onto the stack, the
branch's body is encoded and the condition is popped again.
%
Similarly, the negation of the condition is pushed onto the stack when visiting
an \emph{else} branch, and popped afterwards.
%
Control flow is then encoded using the \emph{if-then-else} operator (ite) as
the SSA $\phi$ function, with the head of the stack as the condition of the
if-then-else expression.

Every arithmetic operation is checked against underflow and overflow according
to the type of the values, and an example is given if there is an underflow or
overflow.
%
We also check whether branch conditions are constant, warning the user about
unreachable blocks or trivial conditions.
%
A \code{require}'s condition is checked for triviality and reachability.
%
This is important since \code{require} conditions are meant to be used as
filters for unwanted input values when they are unknown, for example, in public
functions.
%
Therefore they act like preconditions for the rest of the scope, implied by the
conjunction of the conditions from the conditions stack.
%
Finally, \code{assert} conditions represent target postconditions that the
Solidity programmer wants to ensure at runtime and are verified statically.
%
If it is possible to disprove the assertion, the user is given a
counterexample.

Figure~\ref{figure:solidity_encoding_1} shows a Solidity sample and its SMT
encoding with emphasis on the control flow handling while verifying an assertion.
%
Since variables \code{uint a} and \code{uint b} are function parameters, they
are initialized with the valid range of values for their type (\code{uint256}).
%
The requirement condition about \code{b} is only true if the first condition
about \code{a} is true.
%
The next two assignments to \code{b} create the new SSA variables
$b_1$ and $b_2$.
%
Variable $b_3$ encodes the second and third conditions, and $b_4$
encodes the first condition.
%
Finally, $b_4$ is used in the assertion check.
%
Note that the nested control-flow is implicitly encoded in the $ite$
variables $b_3$ and $b_4$.
%
We can see that the target assertion is safe within its function.

\begin{figure}
\label{figure:solidity_encoding_1}
\noindent\begin{minipage}{.48\textwidth}
\begin{verbatim}
contract C
{
  function f(uint a, uint b) {
    if (a == 0)
      require(b <= 100);
    else if (a == 1)
      b = 1000;
    else
      b = 10000;
    assert(b <= 100000);
  }
}
\end{verbatim}
\end{minipage}\hfill
\begin{minipage}{.48\textwidth}
$a_0 \ge 0 \land a_0 < 2^{256} \land$\\
$(a_0 = 0) \rightarrow (b_0 \le 100) \land$\\
$b_1 = 1000 \land b_2 = 10000$\\
$b_3 = ite(a == 1, b_1, b_2) \land$\\
$b_4 = ite(a == 0, b_0, b_3) \land$\\
$\neg b_4 \le 100000$
\end{minipage}
\caption{SMT encoding of an assertion check highlighting control-flow.}
\end{figure}

It is important to highlight that errors are irrelevant if they result in a
state reversion (Sec.~\ref{section:smart_contracts}). The user is warned
about checks such as overflow only if they do not result in a state reversion.
%
One popular example is the SafeMath~\cite{SafeMath} contract:

\begin{verbatim}
  function add(uint256 a, uint256 b) internal pure returns (uint256) {
    uint256 c = a + b;
    assert(c >= a);
    return c;
  }
\end{verbatim}

Although the tool sees an overflow in the computation of \code{a + b}, since
both have type \code{uint256} which is the largest integer, any execution that
containts the overflow reverts due to the \code{assert}.
%
In this case the user is not warned of the error, since no erroneous cases
exist in accepted executions.

As described above, the component performs several local checks during a single
run, therefore it is critical that the used SMT solver has support to
incremental checking.
%
Moreover, we do not abstract difficult operations such as multiplication
between variables, rather trying to give precise answers when possible.
%
Therefore we combine various quantifier-free theories, such as Linear
Arithmetics, Uninterpreted Functions and Nonlinear Arithmetics. 
%
Solidity has integrated Z3~\cite{Z3} and CVC4~\cite{CVC4} via their C++ APIs.
%
The two SMT solvers are used together to increase solving power.
%
This has been important especially for the programs that require Nonlinear
reasoning, since often one solver is able to prove a property that the other
cannot.
%
In case both Z3 and CVC4's APIs are not available but binaries are, the
component generates \code{smtlib2}~\cite{SMTLIB} formulas and invokes the
solver binaries.


\subsection{Discussions about the future}

\todo{Multi-tx state invariants, with an example, maybe a figure}\\
\todo{Loops?}\\
\todo{effective callback freeness}
