\section{Introduction}
The Ethereum~\cite{WhitePaper} platform is a system that appears as a
singleton networked computer usable by anyone, but is actually built as a
distributed database that utilizes blockchain technology to achieve consensus.
%
One of the features that sets Ethereum apart from other blockchain systems is
the ability to store and execute code inside this database, via the Ethereum
Virtual Machine (\emph{EVM}).
%
In contrast to traditional server systems, anyone can inspect this stored code
and execute functions that can have stateful effects.
%
Since blockchains are typically used to store ownership relations of valuable goods
(for example
cryptocurrencies),
malicious actors have a monetary incentive to analyze the inner
workings of such code. Because of that, testing (i.e.\  dynamic analysis of some
typical inputs) does not suffice and analyzing all possible inputs by utilizing
static analysis or formal verification is recommended.

SAT/SMT-based techniques have been used extensively for program 
verification~\cite{Biere99,Donaldson11,Komuravelli13,Beyer11,Kroening14,Alt17}.
%
This paper shows how the Solidity compiler, which generates EVM bytecode,
utilizes an SMT solver and a Bounded Model Checking~\cite{Biere99} approach to
verify safety properties that can be specified as part of the source code, as
well as fixed targets such as arithmetic underflow/overflow, and detection of
unreachable code and trivial conditions.
%
The main advantage of this system over others for the user is that they do not
need to learn a second verification language or how to use any new tools, since
verification is part of the compilation process.
%
The Solidity language has requirement and assertion constructs that allow to
filter and check conditions at run-time.  The verification component
builds on top of this and tries to
verify at compile-time that the asserted conditions hold for any input,
assuming the given requirements.

Sec.~\ref{section:smart_contracts} introduces the EVM and smart contracts.
Sec.~\ref{section:solidity} gives a very brief overview of Solidity,
the high-level language for smart
contracts. Sec.~\ref{section:smt} discusses the translation from Solidity to SMT
statements and next challenges. Finally, Sec.~\ref{section:conclusion}
contains our concluding remarks.

\begin{paragraph}{Related work.}
Oyente~\cite{Luu2016} and Mythril~\cite{Mythril} are SMT-based symbolic
\todo{there are at least two other porjects to mention, I will check}
execution tools for EVM bytecode that check for specific known vulnerabilities,
where the former also checks for assertion fails. They simulate the virtual
machine and execute all possible paths, which takes a performance toll even
though the approach works well for simple programs.
%
Solidity has been partially translated to the frameworks Why3~\cite{Why3}\todo{We should also cite the documentation} and
F*~\cite{Bhargavan2016}, but the former requires learning a new annotation
specification language and the latter only verifies fixed vulnerability
patterns and does not verify custom user-provided assertions.

\end{paragraph}
