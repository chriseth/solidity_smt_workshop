\section{Smart Contracts}
\label{section:smart_contracts}

Programs in Ethereum are called \emph{smart contracts}.
They can be used to enforce agreements between
mutually distrusting parties as long as all conditions can be fully formalized
and do not depend on external factors. Typical use-cases are decentralized
tokens which can have a currency-like aspect, any mechanisms that build
on top of these tokens like exchanges and auctions or also decentralized
tamper-proof registry systems like a domain name system.

Each smart contract has an \emph{address} under which its \emph{code}
and a key-value store of data (\emph{storage}) is stored. The code is fixed after the creation phase
and only the smart contract itself can modify the data stored at its address.

Users can interact with a smart contract by sending a \emph{transaction}
to its address. This causes the smart contract's code to execute inside
the so-called \emph{Ethereum Virtual Machine} (EVM), which is a stack-based
256-bit machine with a minimalistic instruction set. Each execution environment
has a freshly initialized \emph{memory area} (not to be confused with the persisting
storage). During its execution, a smart contract can also call other
smart contracts synchronously, which causes their code to be run in
a new execution environment. Data can be passed and received in calls.
Furthermore, smart contracts can also create new smart contracts with
arbitrary code.

Since it would otherwise be easy to stall the network by asking it to
execute a complex task, the resources consumed are metered during execution
in a unit called \emph{gas}. Each transaction only provides a certain
amount of gas, and as soon as this \emph{gas limit} is reached without
prior termination, any effect of the transaction is reverted.
In every case, the user
who requested the execution pays for it with Ethereum's native token
Ether proportionally to the amount of gas consumed.

A reverting termination can also happen prior to all gas being consumed. This is
a special feature of the Ethereum Virtual Machine,
which makes the control-flow analysis different from other languages.
Whenever the EVM encounters an invalid situation (invalid opcode, invalid
stack access, etc.), execution will not only stop, but all effects on the
state will be reverted. This reversion takes effect in the current execution
environment, and the environment will also
flag a failure to the calling environment, if present. Typically,
when a call fails, high level languages will in turn cause an
invalid situation in the caller and thus the reversion
affects the whole transaction.

There is also an explicit opcode that causes the current call to fail, which is
essentially the same as described above, but as an \emph{intended} effect.
%
This leads to the important observation that errors might be detected but they
are only relevant if a path that reaches the error does not lead to the
transaction being reverted.

Very briefly, the SMT encoding we will discuss later assumes that no intended
failure happens and tries to deduct that no unintended failure can occur. This
allows the programmer to state preconditions using intended failures and
postconditions using unintended failures.
